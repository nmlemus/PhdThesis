\chapter{Ideas}\label{apendiceB}

\subsection{Variance, Information and Entropy}

\textbf{Variance.} 

\textbf{Information and Entropy.}

\subsection{Information Gain, Distances and Divergences}

\section{Sensitivity Analysis}
Sensitivity analysis is the systematic study of how model inputs—parameters, initial and boundary conditions—affect key model outputs. Depending on the application, one might use local derivatives or global descriptors such as Sobol’s functional decomposition or variance decomposition. Also, the needs of the application may range from simple ranking of the importance of inputs to a response surface model that predicts the output given the input settings. Such sensitivity studies are complicated by a number of factors, including the dimensionality of the input space, the complexity of the computational model, limited forward model runs due to the computational demands of the model, the availability of adjoint solvers or derivative information, stochastic simulation output, and high-dimensional output. Challenges in sensitivity analysis include dealing with these factors while addressing the needs of the application. \cite{DEnergy2009}

\begin{equation}
E=mc^2
\end{equation}

%\lipsum[55-57]
