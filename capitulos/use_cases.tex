\chapter[Use Cases]{Use Cases}\label{cap:use_cases}

In the present chapter we are going to test the UQMS in three different scenarios, spatial only domain, section \ref{Wave Propagation} , spatio-temporal domain, section \ref{spatio_temporal}, and finally a multidisciplinary system, section \ref{NASA}. 

\section{Case Study:  Wave Propagation Problem}\label{Wave Propagation}

The first one is a geophysical tests for wave propagation problems

As a first case study we use the “HPC4E Seismic Test Suite”, a collection of four 3D models and sixteen associated tests that can be downloaded freely at the project's website (https://hpc4e.eu/downloads/datasets-and-software ). The models include simple cases that can be used in the development stage of any geophysical imaging practitioner (developer, tester ...) as well as extremely large cases that can only be solved in a reasonable time using ExaFLOPS supercomputers. The models are generated to the required size by means of a Matlab/Octave script and hence can be used by users of any OS or computing platform. The tests can be used to benchmark and compare the capabilities of different and innovative seismic modelling approaches, hence simplifying the task of assessing the algorithmic and computational advantages that they pose. %\cite{deLaPuente2015}

In our case, we are going to use the “HPC4E Seismic Test Suite” as a case study of the porposed UQMS. As we mention in the introduction of this chapter this model is a spatial only domain problem, because we are going to consider a multidimentional array as an Input and a multidimentional array as an output, but of them time independet.

\subsection{Mathematical Formulation}

\subsection{Model and Dataset Description}
The models have been designed as a set of 16 layers with constant physical properties. The top layer delineates the topography and the other 15 different layer interface surfaces or horizons. In the following, an interface horizon is associated with properties that apply to the layer that exists between itself and the immediately next layer horizon. The model covers an area of 10 x 10 x 5 km, with maximum topography at about 500 m and maximum depth at about 4500 m. The layer horizons have been sampled very finely with 1.6667 m spacing so that a highly accurate representation can be honored at high frequencies. For simulation schemes based on unstructured grids, the layer horizons can be used easily to constrain model blocks. For simulation schemes based upon Cartesian grids, a simple script is provided that can generate 3D grids for any desired spatial sampling. Table \ref{layers_constants} shows the properties of each of the layers included in the models. %\cite{deLaPuente2015}

\begin{table}[]
\centering
\caption{Layer constant properties and their depth range. “Star” layers are only used in the flat case, in substitution of their non-star equivalents}
\label{layers_constants}
\begin{tabular}{|l|l|l|l|l|l|}
\hline
\multicolumn{1}{|c|}{\textbf{\begin{tabular}[c]{@{}c@{}}Layer\\ Id\end{tabular}}} & \multicolumn{1}{c|}{\textbf{\begin{tabular}[c]{@{}c@{}}Vp\\ (m/s)\end{tabular}}} & \multicolumn{1}{c|}{\textbf{\begin{tabular}[c]{@{}c@{}}Vs\\ (m/s)\end{tabular}}} & \multicolumn{1}{c|}{\textbf{\begin{tabular}[c]{@{}c@{}}Density\\ (Kg/m3)\end{tabular}}} & \multicolumn{1}{c|}{\textbf{\begin{tabular}[c]{@{}c@{}}Max. depth\\ (m)\end{tabular}}} & \multicolumn{1}{c|}{\textbf{\begin{tabular}[c]{@{}c@{}}Min. depth\\ (m)\end{tabular}}} \\ \hline
1 & 1618.92 & 500.00 & 1966.38 & -135.55 & -476.35 \\ \hline
2 & 1684.08 & 765.49 & 1985.88 & 41.50 &  -394.90 \\ \hline
3 &  &  &  &  &  \\ \hline
4 &  &  &  &  &  \\ \hline
5 &  &  &  &  &  \\ \hline
6 &  &  &  &  &  \\ \hline
7 &  &  &  &  &  \\ \hline
8 &  &  &  &  &  \\ \hline
9 &  &  &  &  &  \\ \hline
10 &  &  &  &  &  \\ \hline
11 &  &  &  &  &  \\ \hline
12 &  &  &  &  &  \\ \hline
13 &  &  &  &  &  \\ \hline
14 &  &  &  &  &  \\ \hline
15 &  &  &  &  &  \\ \hline
16 &  &  &  &  &  \\ \hline
2* &  &  &  &  &  \\ \hline
3* &  &  &  &  &  \\ \hline
\end{tabular}
\end{table}

\subsection{Adding uncertainty into the model}

The “HPC4E Seismic Test Suite” does not provide uncertainty sources, because all the input parameters of the model have fixed values. Then, to the purpose of our work we need to add some uncertainties into the inputs. Let's suppose the variable ${V_{p}}$ is uncertain. As this variable have 16 different values, one for each layer, we can consider it as a random vector, equation \ref{random_vector_vp}. We associate to each of the $V_{{p}_{i}}$ a Normal distribution with ${\mu}_{i}$ equal to the value reported in Table \ref{layers_constants} and $\sigma=2$.
\begin{equation}\label{random_vector_vp}
V_{p}=<V_{{p}_{i}},\mathcal{N}({\mu}_{i},{\sigma}_{i})>
\end{equation}

\section{Case Study: Austin, queso library}\label{spatio_temporal}

\section{Case Study: Multidisciplinary System (NASA)}\label{NASA}

\section{Case Study: Spatio-temporal Nicholson-Bailey model}
Este esta en el software uqlab, en la carpeta Doc Manuals
