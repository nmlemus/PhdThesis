\chapter[Conclusions and Future Works]{Conclusions and Future Works}\label{cap:conclusions}
Large-scale spatio-temporal simulations produce a huge amount of data that need to be interpreted in order to assess the simulation quality in different regions of space-time. Querying these data poses a great challenge due to their volume and different data distributions. In order to solve this problem, 
in this paper we propose SUQ$^2$, a general approach to answer uncertainty quantification queries. 

The approach uses \textit{GLD} that enables the representation of a spatio-temporal simulation output using a single functional formalism. By modeling each spatio-temporal point by a GLD instance, we can synthesize the region in a number of clusters, represented by their centroid GLD function. From this basis, queries can be answered by combining the centroids in a spatio-time region into GLD-mixture functions. Moreover, by using information entropy techniques, a value can be assigned that represents the uncertainty in a region. The proposed approach is implemented in a workflow that can be extended to solve new UQ queries.

We ran extensive experiments using a seismic use case. The results showed that GLD representation of the data is valid on $85 \%$ of the dataset. Other extensions of the GLD formalism, such as EGLD \cite{Karian2011}, can be evaluated to improve the GLD dataset coverage. Moreover, we showed that the computed centroid function is a good representation of the function instances in its cluster. Additionally, we use the Kolmogorov-Smirnov test to evaluate the quality of the GLD mixture. The p-value, larger than 0.05, assures that the results of the mixture is a good representation of the raw data in the region. Finally, the adoption of the Information Entropy technique was validated by showing the correspondence of the computed values with the uncertainty in the spatio-temporal regions. 

To the best of our knowledge, this is the first work to use GLD as the basis for answering UQ queries in spatio-temporal regions and to compile a series of techniques to produce a query answering workflow.

\section{Revisiting the Research Questions}

\section{Significance and Limitations}

\section{Open Problems and Future Work}
Some of the future directions we are interested in pursuing were mentioned above. For example, in Section \ref{useCaseClustering} we mention that for the purpose of this paper we select \textit{k-means} as the clustering algorithm to be used. This arbitrary selection needs to be studied, and some algorithms implemented to provide an automatic way to cluster the \textit{GLDs}, based on the shapes described in Section \ref{gldShape}.

In Section \ref{useCaseQualityofFit}, there is a region where the \textit{GLD} does not fit well the dataset. If we want to provide a general purpose computational approach for \textit{forward propagation} we need to further investigate this issue.

The use of Information Entropy to quantify the uncertainty is very powerful. However, when applied on clusters of PDFs, such as the GLD, it observes the information variation as a function of the PDF definition, in the case of GLD this is given by its for $\lambda$ parameters. In this context, a complete region modeled by a single GLD function would have a very low information entropy value. This, however would not express the uncertainty modeled by the GLD function, which could be very high. The outcome of the information entropy evaluation must be interpreted by the user. 

\section{Final Considerations}

\subsection*{Acknowledgments}
This work has been funded by CNPq, CAPES, FAPERJ, Inria (SciDISC project) and the European Commission (HPC4E H2020 project) and performed (for E. Pacitti and P. Valduriez) in the context of the Computational Biology Institute (www.ibc-montpellier.fr) and for (F. Porto, H. Lustosa and N. Lemus) in the context of the DEXL Laboratory (dexl.lncc.br)  at LNCC.
