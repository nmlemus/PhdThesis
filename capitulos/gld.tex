\chapter[The Generalized Lambda Distribution]{The Generalized Lambda Distribution}\label{cap:gld}

\begin{flushright}
	\textit{``There are good reasons for using the GLD distribution \\
	methods... GLD fits have been used successfully in many fields ...\\
	Try the GLD first and stop there if the results are acceptable.''\\
	(Karian and Dudewicz, 2011)}
\end{flushright}

\section{The Generalized Lambda Distribution}

\subsection{The Ramberg and Schmeiser Parametrization}
The Generalized Lambda Distribution (GLD) was defined by Ramberg and Schmeiser in 1974 by the quantil function:
\begin{equation}
F^{-1}(p|\lambda)=F^{-1}(p|\lambda_{1}, \lambda_{2}, \lambda_{3}, \lambda_{4})=\lambda_{1}+\frac{p^{\lambda_{3}}-(1-p)^{\lambda_{4}}}{\lambda_{2}}
\end{equation}
where $p$ are the probabilities, $p\in[0,1]$, $\lambda_{1}$ and $\lambda_{2}$ are the location and scale parameteres, and $\lambda_{3}$ and $\lambda_{4}$ determine the skewness and kurtosis of the $GLD(\lambda_{1}, \lambda_{2}, \lambda_{3}, \lambda_{4})$.

Some restrictions in the values of $\lambda_{1}, \lambda_{2}, \lambda_{3}$ and  $\lambda_{4}$ define if the \textit{GLD} is valid. Those restrictions define 6 regions as is shown in table 

% Please add the following required packages to your document preamble:
% \usepackage{multirow}
\begin{table}[]
\centering
\caption{The range of the GLD parameters and the minimum and maximum
values corresponding to the labeling of the regions given in Figure }
\label{my-label}
\begin{tabular}{|c|c|c|c|c|c|c|}
\hline
Region             & $\lambda_{1}$ & $\lambda_{2}$ & $\lambda_{3}$ & $\lambda_{4}$ & Minimum & Maximum \\ \hline
1 and 5            & all            & $< 0$    & $< -1$   & $> 1$ & $-\infty$                             & $\lambda_{1}+\frac{1}{\lambda_{2}}$ \\ \hline
2 and 6            & all            & $< 0$    & $> 1$ & $< -1$   & $\lambda_{1}-\frac{1}{\lambda_{2}}$ & $\infty$                                 \\ \hline
\multirow{3}{*}{3} & all            & $> 0$& $> 0$& $> 0$& $\lambda_{1}-\frac{1}{\lambda_{2}}$ & $\lambda_{1}+\frac{1}{\lambda_{2}}$ \\ \cline{2-7} 
                   & all            & $> 0$& = 0            & $> 0$& $\lambda_{1}$                       & $\lambda_{1}+\frac{1}{\lambda_{2}}$ \\ \cline{2-7} 
                   & all            & $> 0$& $> 0$& = 0            & $\lambda_{1}-\frac{1}{\lambda_{2}}$ & $\lambda_{1}$                       \\ \hline
\multirow{3}{*}{4} & all            & $< 0$    & $< 0$    & $< 0$    & $-\infty$                               & $\infty$                                 \\ \cline{2-7} 
                   & all            & $< 0$    & = 0            & $< 0$    & $\lambda_{1}$                       & $\infty$                                 \\ \cline{2-7} 
                   & all            & $< 0$    & $< 0$    & = 0            & $-\infty$                               & $\lambda_{1}$  
\end{tabular}
\end{table}

The probability density function of the \textit{GLD} at the point $x=F^{-1}(p)$ is given by:
\begin{equation}
f(x)=f(F^{-1}(p))=\frac{\lambda_{2}}{\lambda_{3}p^{\lambda_{3}-1}+\lambda_{4}(1-p)^{\lambda_{4}-1}}
\end{equation}

Note that the valid parameteres of $\lambda$ guaranty that:
\begin{equation}
f(x) \geqslant 0
\end{equation}
\begin{equation}
\int f(x)dx=1
\end{equation}

\subsection{The FMKL Parametrization}

\section{Fitting Mixture Distributions Using a Mixture of Generalized Lambda Distributions}
Esto esta en \cite{Tobergte2013}

\section{From Emperimental Data to GLD Paremeters}
\cite{Lampasi2006}

\section{GLD Shapes}

\section{The GLDEX R package}

\section{GLD mixture}
